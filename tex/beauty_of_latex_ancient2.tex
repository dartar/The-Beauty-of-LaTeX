%!TEX TS-program = xelatex
%!TEX encoding = UTF-8 Unicode

% Dario Taraborelli (2008)
% The Beauty of LaTeX
% URL: http://nitens.org/taraborelli/latex
% Some rights reserved: CC-BY-SA

\documentclass[11pt,a4paper]{article}
\usepackage[dvipdfm, colorlinks, breaklinks, pdftitle={The Beauty of LaTeX},pdfauthor={Taraborelli, Dario}]{hyperref}
\usepackage[usenames]{color}
\definecolor{Gray}{rgb}{.7,.7,.7}
\usepackage{xunicode}
\usepackage{xltxtra}
\defaultfontfeatures{Mapping=tex-text}

\newcommand{\red}[1]{\color{red} #1}
\newcommand{\old}[1]{\fontspec[Alternate=1,Ligatures={Common, Rare}, Swashes={LineInitial, LineFinal}]{Hoefler Text}\fontsize{24pt}{30pt}\selectfont #1}%
\newcommand{\smallprint}[1]{\fontspec{Adobe Garamond Pro}\fontsize{10pt}{13pt}\color{Gray}\selectfont #1}%

\begin{document}
\thispagestyle{empty}
\old\begin{quote}
{\red Q}ue di{\red ct}es vous de mon appel,\\
Garnier ? Fis je sens ou folie ?\\
Toute be{\red st}e garde sa pel\\
{\red Q}ui la contraint, e{\red ff}orce ou lie\\
S'elle peut, elle se deslie
\end{quote}
\vfill{}
\raggedleft\smallprint{D. Taraborelli (2008), \href{http://nitens.org/taraborelli/latex}{The Beauty of \LaTeX}\\\emph{Some rights reserved}. \href{http://creativecommons.org/licenses/by-sa/3.0/}{\textsc{cc-by-sa}}
}
\end{document}
